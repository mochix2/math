\documentclass[a4paper]{bxjsarticle}  
\usepackage{zxjatype}
\usepackage[ipa]{zxjafont}

%usepackage----------
\usepackage{amsthm}
\usepackage{amsmath,mathrsfs,amsfonts,amssymb}
\usepackage{braket} %$\set{|}$, $\Set{|}$
\usepackage{framed} %\begin{leftbar}\end{leftbar}, \begin{framed}\end{framed}

%newcommand-------
%N, Z, Q, R, C
\newcommand{\nat}{\mathbb{N}}
\newcommand{\zah}{\mathbb{Z}}
\newcommand{\quo}{\mathbb{Q}}
\newcommand{\rea}{\mathbb{R}}
\newcommand{\com}{\mathbb{C}}

%絶対値, ノルム
\newcommand{\abs}[1]{\left\lvert#1\right\rvert}
\newcommand{\norm}[1]{\left\lVert#1\right\rVert}

%記号
\newcommand{\ds}{\displaystyle}
\newcommand{\Forall}{{}^{\forall}}
\newcommand{\Exists}{{}^{\exists}}
\newcommand{\Existsonly}{{}^{\exists !}}

%theorem-------------
\theoremstyle{definition}
\newtheorem{thm}{定理}[section]
\newtheorem{prop}[thm]{命題}
\newtheorem{defn}[thm]{定義}
\newtheorem{lemm}[thm]{補題}
\newtheorem{cor}[thm]{系}
\newtheorem{exm}[thm]{例}
\newtheorem{rem}[thm]{注意}
\newtheorem{axiom}[thm]{公理}
\newtheorem{thms}[thm]{定理図式}

%--------------------

\title{数学の基礎}
\author{nessinverse}
\date{\today}

\begin{document}
    \maketitle
    
    \section*{まえがき}
    \subsection*{内容}
    ZFCにより展開される, 公理的集合論の初歩について扱う. 
    % もっと詳しい内容
    \subsection*{モチベーション}
    % 語る
    
    \subsection*{前提知識}
    \cite{matsuzaka}程度の素朴集合論. すなわち, 素朴な集合, 関数, 関係, 順序の概念について既知であることが望ましい.
    \subsection*{記法}
    \begin{itemize}
        \item \textbf{ON}のように, クラスは太字で記述する. 通常の字体で記述されているものは集合である.
        \item インフォーマルな議論, すなわち「気持ち」を述べた正確でない議論には下線をつける.
        
        \underline{これはインフォーマルな議論です.}
    \end{itemize}
    \newpage
    \tableofcontents
    \newpage    
    
    \section{数学の公理化}
    % そもそも, 数学とはどのような行為かを思い返してみると,
    % \begin{itemize}
    %     \item 数学的な対象を定義する
    %     \item 定義された対象が満たす, 定理を証明する
    % \end{itemize}
    % ことの繰り返しでした. 証明には一切の曖昧さを排し, 定義やこれまでに示した定理しか用いてはいけません. しかし素朴集合論の教えるところでは, そもそも集合の定義が厳密には与えられていませんでした.\\
    % 今回は完全に公理化された集合論の体系を構築することで, 曖昧さを一切取り除くことにしましょう. この方法で数学は, 集合とは何か, 元が集合に属するとは何かについて言及することなしに証明という手続きによって定式化されます.
    
    \subsection{式と文}
    \begin{defn}
        次に列挙するものを総称して\textbf{記号}という.
        \begin{enumerate}
            \item $\forall$ 全称量化
            \item $\exists$ 存在量化
            \item $\land$ 論理積
            \item $\lor$ 論理和
            \item $\lnot$ 論理否定
            \item $\to$ 含意
            \item $=$ 等号
            \item $\in$ 所属
            \item  ( \,\,左括弧
            \item  ) 右括弧
            \item $x_1, x_2, \dots$ 変数
        \end{enumerate}
        \textbf{量化子}と言えば$\forall, \exists$, \textbf{結合子}と言えば$\land, \lor, \lnot, \to$, \textbf{述語}と言えば$=, \in$, \textbf{括弧}と言えば(, )を表す.
    \end{defn}
    \begin{defn}    
        次の規則によって再帰的に定まる記号の列を\textbf{式}という.
        \begin{enumerate}
            \item 「変数, 述語, 変数」の形の記号列. (原子式)
            \item 「(, 式, ), 結合子, (, 式, )」の形の記号列. (式の結合)
            \item 「量化子, 変数, (, 式, )」の形の記号列. (式の量化)
        \end{enumerate}
        式が変数から始まっているか, 左括弧から始まっているか, 量化子から始まっているかに注目すると, 式が2通り以上のやり方で形成されることはないことがわかる. この一意的な式の形成の途中の式のことを, 元の式の\textbf{部分式}という.
    \end{defn}
    \begin{defn}
        式$F$の中の量化子$Q$の出現位置について, その\textbf{量化変数}とは, その位置の直後に出現する変数のことをいう.  その\textbf{スコープ}とは, 式$F$の形成において, その$Q$が量化の手順で導入されたときの, 「$Q$, 変数, (, 式, )」なる$F$の部分式のこと.
    \end{defn}
    \begin{defn}
        式の中の変数$x$の出現位置について,  それが\textbf{束縛されている}とは, $x$を量化変数とする, ある量化子のスコープにあることをいう. それが\textbf{自由である}とは, 束縛されていないことをいう.\\
        式が\textbf{文}であるとは, その式の中の全ての変数が自由であることをいう.
    \end{defn}
    \begin{rem}
        出現位置という言葉を用いたのは, 式に複数回同じ量化子や変数が現れてもそれを区別できるようにするためである.\\
        ゆえにスコープとは, 式に量化子が出現するたびに個別に定まる概念であることに注意せよ. 例えば, 式$\forall x_1 ((x_1=x_2) \land (\forall x_2 (x_2 \in x_3)))$ において1つ目の全称量化子のスコープはこの式全体で, 2つ目の全称量化子のスコープは$\forall x_2 (x_2 \in x_3)$ である.\\
        また, 束縛されていること, 自由であることも式に変数が出現するたびに定まる概念であることに注意せよ. 例えば, 今の例では1つ目の$x_2$は自由だが, 2つ目と3つ目の$x_2$は束縛されている.
    \end{rem}
    \subsection{論理の公理}
    \begin{defn}
        文が\textbf{等号の公理}であるとは, 変数$x,y,z,w$によって以下の形をしていることをいう.
        \begin{itemize}
            \item $\forall x (x=x)$
            \item $\forall x (\forall y ((x=y) \to (y=x)))$
            \item $\forall x (\forall y (\forall z (((x=y) \land (y=z)) \to (x=z))))$
            \item $\forall x (\forall y (\forall z (\forall w ((((x=y)\land(z=w))\to((x\in z)\to(y\in w)))))))$
            
            
            % x,y,z,wを入れ替えた24通り必要?
            % i.e. \forallの順序が入れ替えるとnontriv.
        \end{itemize}
    \end{defn}
    \begin{defn}
        文が\textbf{命題論理のトートロジー}であるとは, 次のように定義される.\\
        
    \end{defn}
    
    
    % \begin{rem}
    % ここまでで, \begin{itemize}
    %     \item 記号の定義に, 変数が「\underline{無限個}」あること
    %     \item 式の定義の「再帰的」などといった言い回し
    % \end{itemize}
    % に厳密でないような気分を感じた人がいるかもしれません.
    % しかし, 私たちは\underline{式全体の集まり}を定義したのではなく, \underline{式の形成アルゴリズム}を定義したことに注意してください.\\ 与えられた記号列が式であるかを判定できて, それさえできれば十分です.
    % \end{rem}
    
    
    \section{順序数論}
    \section{基数論}
    \section{応用}
    % Zorn
    % algebraic closure
    % Measure theory
    % Category Theory
    \begin{thebibliography}{9}
    \bibitem{matsuzaka} 松坂和夫, 『集合・位相入門』, 岩波書店, 1968.
    
    \end{thebibliography}
\end{document} 