\documentclass[a4paper]{bxjsarticle}  
\usepackage{zxjatype}
\usepackage[ipa]{zxjafont}

%usepackage----------
\usepackage{amsthm}
\usepackage{amsmath,mathrsfs,amsfonts,amssymb}
\usepackage{braket} %$\set{|}$, $\Set{|}$
\usepackage{framed} %\begin{leftbar}\end{leftbar}, \begin{framed}\end{framed}
%
%newcommand-------
%N, Z, Q, R, C
\newcommand{\nat}{\mathbb{N}}
\newcommand{\zah}{\mathbb{Z}}
\newcommand{\quo}{\mathbb{Q}}
\newcommand{\rea}{\mathbb{R}}
\newcommand{\com}{\mathbb{C}}

%絶対値, ノルム
\newcommand{\abs}[1]{\left\lvert#1\right\rvert}
\newcommand{\norm}[1]{\left\lVert#1\right\rVert}

%記号
\newcommand{\ds}{\displaystyle}
\newcommand{\Forall}{{}^{\forall}}
\newcommand{\Exists}{{}^{\exists}}
\newcommand{\Existsonly}{{}^{\exists !}}

%theorem-------------
\theoremstyle{definition}
\newtheorem{thm}{定理}[section]
\newtheorem{prop}[thm]{命題}
\newtheorem{defn}[thm]{定義}
\newtheorem{lemm}[thm]{補題}
\newtheorem{cor}[thm]{系}
\newtheorem{exm}[thm]{例}
\newtheorem{rem}[thm]{注意}
\newtheorem{axiom}[thm]{公理}
\newtheorem{thms}[thm]{定理図式}

%--------------------

\title{数学の基礎}
\author{nessinverse}
\date{\today}

\begin{document}
    \maketitle
    
    \section*{まえがき}
    \subsection*{内容}
    ZFCにより展開される, 公理的集合論の初歩について扱う. 
    % もっと詳しい内容
    \subsection*{モチベーション}
    % 語る
    
    \subsection*{前提知識}
    \cite{matsuzaka}程度の素朴集合論. すなわち, 素朴な集合, 関数, 関係, 順序の概念について既知であることが望ましい.
    \subsection*{記法}
    \begin{itemize}
        \item \textbf{ON}のように, クラスは太字で記述する. 通常の字体で記述されているものは集合である.
        \item インフォーマルな議論, すなわち「気持ち」を述べた正確でない議論には下線をつける.
        
        \underline{これはインフォーマルな議論です.}
    \end{itemize}
    \newpage
    \tableofcontents
    \newpage    
    
    \section{数学の公理化}
    \section{順序数論}
    \section{基数論}
    \section{応用}
    % Zorn
    % algebraic closure
    % Measure theory
    % Category Theory
    \begin{thebibliography}{9}
    \bibitem{matsuzaka} 松坂和夫, 『集合・位相入門』, 岩波書店, 1968.
    
    \end{thebibliography}
\end{document}